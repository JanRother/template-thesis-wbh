\chapter{Beispielkapitel}
\label{ch:01_example}

\begin{comment}
Dies ist ein mehrzeiliger Kommentar.
Er wird nicht im Dokument angezeigt.
\end{comment}

\section{Eine Beispielsektion}
\label{sec:01-00_example}

\subsection{Subsektion}
\label{subsec:01-00-00_subsection}

\enquote{Das ist ein Beispielzitat.} \footnote{Fußnote}

Das Dokument ist in \Gls{latex} geschrieben.
Texte können mit einer \gls{ide} bearbeitet werden.
Eine \gls{abk} ist möglich und kann als \gls{abk} mehrfach verwendet werden.

\lipsum[1-3]

\autoref{fig:p-00_example-picture} zeigt ein \nameref{fig:p-00_example-picture}.

\image{H}{\defaultImageSize}{p-00_example-picture}{Beispielbild}

\lipsum[4-5]

Der folgende Quellcode ist ein Beispiel für eine Codebox:

\begin{snippet}[H]
    \centering
    \begin{codebox}{javascript}{\texttt{Example Code}}
		function example() {
			console.log('This is an example code snippet.');
		}
    \end{codebox}
    \caption{\texttt{Example Code} als Auszug aus einer Software}
    \label{snippet:01-00-00_example-code}
\end{snippet}

\lipsum[6-7]

Eine Tabelle fässt wichtige Inhalte zusammen.

\begin{table}[H]
    \centering
    \begin{university-table}{width=\textwidth, colspec={l X[1,l] X[2,l] c}}
        \textbf{Element} & \textbf{Beschreibung} & \textbf{Liste} & \textbf{Abb.} \\
        \textbf{Beispiel} & 
            Ein einfaches Beispiel. & 
            \begin{itemize}[before=\vspace{-1em}]
                \item ein Punkt
				\item noch ein Punkt
				\item und ein weiterer Punkt
            \end{itemize} & 
            \ref{fig:p-00_example-picture} \\
        \textbf{Weiteres} & 
			Ein weiteres Beispiel. & 
			 & 
			n/a \\
    \end{university-table}
    \caption{Elemente mit Beschreibung und Liste in tabellarischer Form}
    \label{tab:01-00-00_example-table}
\end{table}

\subsubsection{Subsubsektion}
\label{subsubsec:01-00-00-00_subsubsection}

\lipsum[8-9]

Hier folgt eine Aufzählung von Optionen:

\begin{itemize}[label={}, rightmargin=2.5cm]
    \item \textbf{Option A}	\dotfill	Dies ist die erste Option.
    \item \textbf{Option B}	\dotfill	Dies ist die zweite Option.
    \item \textbf{Option C}	\dotfill	Dies ist die dritte Option.
\end{itemize}

\subsection{Fazit}
\label{subsec:01-00-01_conclusion}

\lipsum[10]

Das ist ein Fazit aus einer Referenz. \cite{000:Reference}

\lipsum[11]

\section{Eine weitere Beispielsektion}
\label{sec:01-01_example}

\subsection{Unterpunkt}
\label{subsec:01-01-00_subsection}

\lipsum[12-13]

\subsection{Fazit}
\label{subsec:01-01-01_conclusion}

\lipsum[14]
