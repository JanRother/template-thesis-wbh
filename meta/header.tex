% PAGE LAYOUT



\selectlanguage{ngerman}								% set language to german

\setlength{\parindent}{0em} 							% paragraph indentation to left-justified

\onehalfspacing											% set line spacing to 1.5

\makeatletter

% configure headers
\def\maxChapterTitleLength{44}							% define maximum length of chapter title in header

\patchcmd{\@makechapterhead}{\vspace*{50\p@}}{}{}{} 	% removes space above \chapter head
\patchcmd{\@makeschapterhead}{\vspace*{50\p@}}{}{}{} 	% removes space above \chapter* head
\makeatother
\setlength{\headheight}{14.5pt}							% set header height
\pagestyle{fancy}										% set page style to fancy

\renewcommand{\chaptermark}[1]{
	\noexpandarg
	\StrLen{#1}[\chapterTitleLength]

	\ifthenelse{\chapterTitleLength > \maxChapterTitleLength}{
		\StrLeft{#1}{\maxChapterTitleLength}[\title]
        \edef\title{\title \ldots}
	}{
		\def\title{#1}
	}

	\markboth{\title\ -- \chaptername\ \thechapter\ }{}	% set chapter mark
}

\renewcommand{\sectionmark}[1]{							% set section mark
	\markright{
		{\projectShortDescription}
	}
}

\fancypagestyle{titlepage}								% set title page style
{
	\setcounter{page}{-100000}
	\fancyhf{}
	\fancyfootoffset{0pt}
	\fancyheadoffset{0pt}
	\renewcommand{\headrulewidth}{0pt}
}

% set fonts of titles
\titleformat{\chapter}[display] {\sffamily \huge}{\chaptertitlename\ \thechapter}{-5pt}{\Huge}
\titlespacing*{\chapter}{0pt}{0pt}{10pt}
\titleformat{\section}[display] {\sffamily \tiny}{}{0pt}{\LARGE \thesection\ }
\titlespacing*{\section}{0pt}{0pt}{0pt}
\titleformat{\subsection}[display] {\sffamily \tiny}{}{-15pt}{\Large \thesubsection\ }
\titlespacing*{\subsection}{0pt}{0pt}{0pt}
\titleformat{\subsubsection}[display] {\sffamily \tiny}{}{-15pt}{\large \thesubsubsection\ }
\titlespacing*{\subsubsection}{0pt}{0pt}{0pt}

% set fonts of table of contents
\renewcommand{\cftchapfont}{\bf\sffamily}
\renewcommand{\cftsecfont}{\sffamily}
\renewcommand{\cftsubsecfont}{\sffamily}

% configure glossary and acronyms
% -> styles: https://www.dickimaw-books.com/gallery/glossaries-styles/
\setglossarystyle{altlistgroup}
\setacronymstyle{long-short}

% set fonts of footnotes
\renewcommand{\cfttabfont}{\sffamily}
\renewcommand{\cftfigfont}{\sffamily}

\setcounter{secnumdepth}{3}

% set metadata of document
\hypersetup{
	pdfauthor={\firstAuthor},
	pdftitle={\documentTitle\ -- \documentSubtitle},
	pdfsubject={\documentSubject \documentYear},
	pdfkeywords={Betreut von \tutorAName und \tutorBName.}
}

% set code listings
\lstset{
	showspaces=false,
	showtabs=false,
	breaklines=true,
	showstringspaces=false,
	basicstyle=\ttfamily,
	frame=lt,
	rulecolor=\color{gray},
	framerule=3pt,
	xleftmargin=6pt,
}

% custom colors
\definecolor{whiteSmoke}{HTML}{F5F5F5}
\definecolor{universityBlue}{HTML}{272f82}
\definecolor{universityPink}{HTML}{DA005F}

% set code boxes
\newtcblisting[auto counter]{codebox}[3][]{%
    listing engine=minted,
    listing remove caption=false,
    % available styles: https://www.overleaf.com/learn/latex/Code_Highlighting_with_minted#Reference_guide
    minted style=manni,
    minted language=#2,
    minted options={tabsize=2,breaklines,autogobble,linenos,numbersep=3mm,escapeinside=||, stripnl=false,breakafter=-/\_()},
    colback=white,
    colframe=universityBlue,
    listing only,
    left=6mm,
    lefttitle=0mm,
    title=\faCode\hspace{0.25em} #3,
    enhanced,
    overlay={
        \begin{tcbclipinterior}
            \fill[whiteSmoke](frame.south west)
            rectangle ([xshift=6mm]frame.north west);
        \end{tcbclipinterior}
    },
    #1
}

% define new caption type for code snippets
\DeclareCaptionType{snippet}[Codeausschnitt][Verzeichnis der Codeausschnitte]

% counter for tracking points
\newcounter{universityTaskPoints}
\newcounter{universityTaskCounter}

% command to convert simple points to points with leading zero and correct singular or plural form
\newcommand{\universityFormatPoints}[1]{
	\ifnum#1<10
		\ifnum#1=1
			0#1 Punkt
		\else
			0#1 Punkte
		\fi
	\else
		#1 Punkte
	\fi
}

% command to add points for subtasks
\newcommand{\universityPoints}[1]{
	\addtocounter{universityTaskPoints}{#1}
	\addtocounter{universityTaskCounter}{1}
	\par\vspace{0.5em}
	\begin{flushright}
		\textit{\textbf{\universityFormatPoints{#1}}}
	\end{flushright}
}

% task environment
\newtcolorbox{universityTask}[1][]{
	colback=gray!5!white,
	colframe=universityPink,
	lefttitle=0mm,
	title=\faCheckSquareO\hspace{0.25em} Aufgabenstellung, 
	fonttitle=\bfseries,
	before upper={\setcounter{universityTaskPoints}{0}\setcounter{universityTaskCounter}{0}},
	after upper={
		\ifnum\value{universityTaskCounter}>0													% change to 1, to only show sum when real sub-tasks exist
			\begin{flushright}
				\par\noindent\makebox[\linewidth][r]{\rule{2.5cm}{0.6pt}}\par\vspace{0.5em}
				\textit{\textbf{$\Sigma$ \universityFormatPoints{\arabic{universityTaskPoints}}}}
			\end{flushright}
		\fi
	},
	#1
}

% style captions
\newlength\myx
\setlength\myx{\textwidth}
\addtolength\myx{-2\fboxsep}
\DeclareCaptionFont{white}{\color{white} \sffamily}
\DeclareCaptionFormat{listing}{\colorbox{gray}{\parbox{\myx}{#1#2#3}}}
\captionsetup[lstlisting]{format=listing,labelfont=white,textfont=white}
\renewcommand{\lstlistingname}{Quelltext}